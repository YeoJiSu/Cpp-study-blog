\documentclass[11pt, a4paper]{article}
\usepackage[hangul]{kotex}
\usepackage{csquotes}
\usepackage{indentfirst}\setlength\parindent{2em}

\begin{document}

\title{과제3 \LaTeX을 활용한 간단한 문서(article) 작성}
\author{한상곤\footnote{연락은 sangkon@pusan.ac.kr}}
\date{2022.09.28}
\maketitle

\begin{displayquote}
“The only way to learn a new programming language is by writing programs in it.”  - Dennis Ritchie{\footnote{https://en.wikipedia.org/wiki/Dennis\_Ritchie}}
\end{displayquote}

\section{목표}
\LaTeX를 사용해서 간단한 문서를 작성하는 것이 과제3의 목표입니다. \LaTeX은 기존의 문서 편집기와 다른 관점을 요구하기 때문에 꾸준한 연습이 필요합니다. \LaTeX을 처음 학습하는 분들이 이번 과제를 통해서 \LaTeX에 익숙해지길 기대합니다.

\section{안내}
\subsection{진행 방법}
과제1의 내용을 \LaTeX을 활용해서 간단한 문서를 작성하시면 됩니다. 그리고 과제3 작성시 기존의 과제1에서 미흡했던 부분을 보완해서 문서를 작성하시면 됩니다. 해당 과제는 \LaTeX에 익숙해지는 것이 가장 중요한 목표이기 때문에 기존 과제를 활용하시고, 부족한 부분은 보완하세요.

\subsection{과제3 주의사항}
\begin{itemize}
    \item 참고문헌(URL, 책 등)은 반드시 각주에 작성하세요.
    \item 참고문헌은 아래 형식을 반드시 참고해서 작성하세요.
    \begin{itemize}
        \item 김도형, 파이썬 기반 금융 인공지능, 한빛미디어, 2022.
        \item S. Marlow, S. P. Jones, and S. Singh., “Runtime support for multicore Haskell,” ACM SIG PLAN Notices. ACM,2009.
        \item M. Crawford, "Catching the Sun," asme.org, \\ https://www.asme.org/engineering-topics/articles/ \\ renewable-energy/catching-the-sun (accessed Feb. 5, 2022).
        \item "Engineering Triumph That Forged a Nation: Panama Canal Turns 100,"  msnbc.msn.com, http://www. \\ msnbc.msn.com/news/world/engineering-triumph-forged-nation-panama-canal-turns-100-n181211 (accessed Nov. 3, 2021).
    \end{itemize}
\end{itemize}

\subsection{제출 관련}
\begin{itemize}
    \item 제출은 tex 파일로 제출하세요(PDF 제출하지 않으셔도 됩니다).
    \item 파일명은 '홍길동-공학작문-03.tex' 입니다.
    \item 제출은 PLATO에서 하시고, 2022년 10월 4일 화요일 저녁 11시까지 입니다.
\end{itemize}

\section{표절 관련 및 주의}
\LaTeX{} 관련 파일을 제출하실 때는 반드시 해당 파일이 컴파일 되는지 확인하세요. 만약 \LaTeX 파일이 컴파일 되지 않으면 과제 점수에 반영되지 않습니다. 꼭 유의하세요. 

과제물을 무단으로 표절하는 것은 어떤 이유에서든 허용되지 않습니다. 표절은 의도와 무관하게 부정 행위로 간주됩니다. 완성된 결과물을 주고받는 것에 유의하세요. 관련 학생 모두 0점/F학점 처리됩니다.

\end{document}